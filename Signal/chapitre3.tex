\section{Systèmes linéaires-Filtres}
\subsection{Définition d'un SL}
$x(t)\xrightarrow{S}y(t)$ avec $x(t)$ et $y(t)$ signal de sortie. On a une relation entrée/sortie. \\
$\forall x(t)$, en supposant connaître S, on sait déterminer la sortie.\\
$\mathcal{L}\{x(t)\} = y(t)$ : L'opérateur du système appliqué au signal d'entrée donne l'observation.

\paragraph{Ensemble des propriétés :}
\begin{itemize}
\item Linéaire : \[\forall \alpha \beta,\ \alpha x_1(t)+\beta x_2(t) \xrightarrow{S} \alpha y_1(t) + \beta y_2 (t)\]
\item Invariance temporelle (ou par translation) : \[x(t-\theta) \xrightarrow{S} y(t-\theta)\]
\item Instantané ou sans mémoire $y(t_1)$ ne dépend que de $x(t_1)$, et pas de son passé.
\end{itemize}

\subsection{Filtre linéaire}
\begin{Def}
Un filtre linéaire (FL) est un système linéaire invariant par translation (SLIT)
\end{Def}

\begin{Def}
Un FL est un convoluteur.
\end{Def}
On note h(t) la fonction qui caractérise le système, appelée réponse impulsionnelle (RI).
\[y(t)=x(t)*h(t)\]

\begin{Def}
Les signaux exponentiels sont les signaux propres du SL.
\end{Def}

On caractériseles FL par leur réponse impulsionnelle. Si on ne connaît pas la RI, on met la Dirac en entrée. Ainsi : \[y(t)=\delta(t)*h(t)=h(t)\]

\begin{Def}
On pose la réponse en fréquence du filtre : \[H(f)=|H(f)|e^{-j\phi(t)}=\int_{-\infty}^{+\infty} h(t)e^{-2\pi jft} dt\]
\end{Def}

\subsection{Filtre en parallèle/En série}
\subsubsection{En série :}
Filtre équivalent : $h(t)=h_1(t)*h_2(t)$ \\
Réponse en fréquence : $H(f)=H_1(f)\times H_2(f)$

\subsubsection{En parallèle :}
Filtre équivalent : $h(t)=h_1(t)+h_2(t)$ \\
Réponse en fréquence : $H(f)=H_1(f)+H_2(f)$

\subsection{Filtre réalisable physiquement}
\begin{Def}
Un FL est réalisable physiquement s'il est stable et causal.
\end{Def}

\ul{Causal :} La sortie ne peut précéder l'entrée.
\[y_k=\sum_{m=-\infty}^k x_m h_{k-m}=\sum_{m=0}^{+\infty}h_m x_{k-m}\]
On a donc $\left\{ \begin{array}{r c l}
h_k\neq 0 &\text{ pour } k\geq0\\
h_k=0 &\text{ pour } k<0
\end{array}
\right.$

\subsubsection{Stabilité BIBO / EBSB}
\noindent BIBO : Bounded Input - Bounded Output \\
EBSB : Entrée Bornée - Sortie Bornée.

\bigskip
\[M=\sum |x_k| <\infty\]
\[\sum h_k x_{k-m} < M\sum |h_k|\]
\begin{theo}
\[\sum h_k <\infty \Leftrightarrow \{|z|=1\}\in \text{ région de convergence}\]
\end{theo}

Pour avoir un filtre physiquement réalisable, on doit avoir tous les pôles à l'intérieur du cercle unité. \\
(Logique, la région de convergence pour les signaux causaux sont à l'extérieur du cercle défini par les pôles, et on doit avoir le cercle unité dans la région de convergence)

\subsection{Filtre Passe-bas / Passe-haut / Passe-bande}

\noindent Voir cours et dessins. \\
Penser que les filtres sont toujours symétriques par rapport à l'axe des ordonnées. 

\subsection{Filtres RIF / RII}
On pose : \[H(z)=\frac{\sum_{i=0}^N b_i z^{-i}}{\sum_{k=0}^M a_k z^{-k}} \text{ avec } a_0=1\]
\subsubsection{RIF : Réponse Impulsionnelle Finie}
$\underbrace{D(z)}_{\text{Dénominateur}}=1 \Rightarrow \text{Filtre stable}$

Ici, on a : \[y_k = b_0 x_k + \underbrace{b_1 x_{k-1} + ... + b_N x_{k-N}}_{\text{Passé}}\]
Dans les RIF, la sortie ne dépend que du passé de l'entrée.

\subsubsection{RII : Réponse impulsionnelle infinie :}
\ul{Premier cas :}\[H(z)=\frac{1}{\sum_{k=0}^M a_k z^{-k}}\]
\begin{eqnarray*}
Y(z)\times \sum_{k=0}^M a_k z^{-k} = X(Z) \xrightarrow{TZ^{-1}}& & \sum_{i=0}^M a_i y_{k-i}=x_k \\
& & y_k = x_k - \sum_{i=1}^M a_i y_{k-i}
\end{eqnarray*}

Filtre purement récursif : la sortie dépend de son propre passé, et de celui-ci uniquement.

\ul{Deuxieme cas :}
\begin{eqnarray*} 
Y(z)\times \sum_{k=0}^M a_k z^{-k} &=& X(Z) \sum_{i=0}^N b_i z^{-i} \\
\xrightarrow{TZ^{-1}} \sum_{i=0}^M a_i y_{k-i} &=& \sum_{l=0}^N b_l x_{k-l} \\
y_k = \sum_{l=0}^N b_l x_{k-l} &-& \sum_{i=1}^M a_i y_{k-i}
\end{eqnarray*}

\noindent $\Rightarrow$ Dépend du passé de l'entrée et de la sortie. \\
Filtre récursif (mais pas purement).
