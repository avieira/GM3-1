\section{Signaux à temps discret}
\subsection*{Introduction}
Jusqu'à présent, nous n'avions qu des signaux continus. On s'intéresse à présent à des signaux à temps discret.
\[x_k, k\in\mathbb{Z}, temps\ discret\ TD\]

Par rapport à un signal continu, on peut poser un pas d'échantillonnage T qui nous permettra de discrétiser notre signal.
\[x_k=x(kT)\]
$x_k$ sera l'échantillon. On a donc besoin d'un infinité d'échantillon pour retrouver le signal continu.

De même, on peut redéfinir plusieurs notions du cas continu dans le cas discret : \\
\begin{tabular}{|c|c|}
\hline
\begin{bf}Cas continu\end{bf} & \begin{bf}Cas discret \end{bf} \\
\hline
$E_x = \int |x(t)|^2 dt$ & $E_x = \sum_{k=-\infty}^{+\infty} |x_k|^2$ \\
\hline
$x(t)*y(t)=\int x(\tau)y(t-\tau)d\tau$ & $x_k*y_k=\sum_{m=-\infty}^{+\infty} x_m y_{k-m}$ \\
\hline
$u(t)=\left\{ \begin{array}{l} 1\ pour\ t \geq 0\\ 0\ sinon \end{array} \right.$ & $u_k=\left\{ \begin{array}{l} 1\ pour\ k \geq 0\\ 0\ sinon \end{array} \right.$\\
\hline
$\delta(t) = \left\{ \begin{array}{l} 1\ si\ t = 0\\ 0\ sinon \end{array} \right.$ & $\delta_k = \left\{ \begin{array}{l} 1\ si\ k= 0\\ 0\ sinon \end{array} \right.$ \\
\hline
\end{tabular}

\subsection{Transformée en z, $z \in \mathbb{C}$}
\{$x_k$\} $\xrightarrow{TZ}$ X(z) = $\underset{|r_1|<|z|<|r_2|}{\sum_{k=-\infty}^{+\infty} x_k z^{-k}}$ : on précise toujours un domaine de convergence au sens du critère de Cauchy.

\begin{bf} Critère de Cauchy :\end{bf} $\sum_{k=0}^{+\infty} u_k$ converge ssi $\lim_{k \rightarrow +\infty} (u_k)^{\frac{1}{k}} <1$
\[X^+ (z) = \sum_{k=0}^{+\infty} x_k z^{-k}\]
\begin{eqnarray*}
\lim_{k \rightarrow +\infty} |u_k z^{-k}|^{\frac{1}{k}}&<&1 \\
\lim_{k \rightarrow +\infty} |u_k|^{\frac{1}{k}} |z^{-1}| &<& 1 \\
r_1 |z^{-1}| &<& 1 \\
r_1 &<& |z|
\end{eqnarray*}

Si $\sum_{k=-\infty}^{+\infty}$ : bilatérale. Si $\sum_{k=0}^{+\infty}$ : monolatérale.

\subsubsection*{Quelques définitions}
\begin{itemize}
\item Un signal $x_k$ est \ul{causal} s'il est nul pour k<0. Dans ce cas, X(z) est monolatéral et la région de convergence est supérieure à $r_1$
\item Un signal $x_k$ est \ul{anticausal} s'il est nul pour $k\geq0$. Dans ce cas, X(z) est monolatéral et la région de convergence est inférieure à $r_2$
\item Un signal $x_k$ est \ul{bilateral} s'il est défini non nul pour tout k. Le domaine de convergence est compris entre $r_1$ et $r_2$.
\end{itemize} 

\begin{bf} Attention ! \end{bf} :
\begin{eqnarray*}
a^k u_k &\xrightarrow{TZ}& \frac{1}{1-az^{-1}} pour |z| > |a| \\
-a^k u_{-k-1} &\xrightarrow{TZ}& \frac{1}{1-az^{-1}} pour |a| > |z|
\end{eqnarray*}

\subsubsection*{Propriétés}
\begin{itemize}
\item $\delta_k \xrightarrow{TZ} 1$
\item $x_k \xrightarrow{TZ} X(Z)$, on pose $y_k = x_{k-m}$ 
\[Y(Z) = \sum_{k=-\infty}^{+\infty} y_k z^{-k} \sum_{k} x_{k-m} z^{-k}\]
On pose l = k-m.
\[Y(z) = \sum_{l} x_{l} z^{-l-m} = z^{-m} X(z)\]
\item $x_k * y_k \xrightarrow{TZ} X(z)Y(z)$ \\
On a vu en TD que $x_k *z^k = z^k X(z)$. On pose $y_k = x_k * h_k$. On a $y_k * z^k = Y(z)$ \\
\[(x_k * h_k)* z^k = x_k * (h_k * z^k) = x_k * (H(z)z^k) = (x_k * z^k)H(z) = z^k X(z)H(z)\]
Par identification, $Y(z)=X(z)Y(z)$.
\item $kx_k \xrightarrow{TZ} -z \frac{dX(z)}{dz}$ \\
On prend $x_k$ et sa TZ $X(Z)=\sum_k x_k z^{-k}$. On pose $y_k = kx_k$. Ainsi, $Y(z)=\sum_k kx_kz^{-k}$. Or, 
\[\frac{dX(z)}{dz}=-\sum_k kx_k z^{-k-1} = -z^{-1} Y(z)\]
D'où le résultat. \\
On peut même aller jusqu'à la dérivée seconde. 
\[\frac{d^2 X(z)}{dz^2}=\sum_k k(k+1)x_k z^{-k-2} = -z^{-2} \left(\sum_k k^2 x_k z^{-k} + \sum_k kx_k z^{-k} \right)\]
Si on pose $w_k=k^2x_k$ on a : 
\[W(z) = z^2\ \frac{d^2 X(z)}{dz^2} + z\ \frac{dX(z)}{dz}\]
\end{itemize}

\subsection{Transformée inverse}
Commençons par un exemple : \[X(z) = \frac{1}{(1-\frac{1}{2} z^{-1})(1-\frac{1}{3} z^{-1})}=\frac{3}{1-\frac{1}{2} z^{-1}} - \frac{2}{1-\frac{1}{3} z^{-1}}\]
Les pôles sont les valeurs qui annulent le dénominateur. \[p_1 = \frac{1}{2}\ et\ p_2 = \frac{1}{3}\]
$\frac{3}{1-\frac{1}{2} z^{-1}}$ correspondra au signal $\frac{1}{2}^k u_k$ si $|z| > \frac{1}{2}$ et à $-\frac{1}{2}^k u_{-k-1}$ si $|z| < \frac{1}{2}$. \\
$\frac{-2}{1-\frac{1}{3} z^{-1}}$ correspondra au signal $\frac{1}{3}^k u_k$ si $|z| > \frac{1}{3}$ et à $-\frac{1}{3}^k u_{-k-1}$ si $|z| < \frac{1}{3}$. \\
Ainsi, pour avoir un signal causal, on prend $|z| > \frac{1}{2}$ et ainsi, $x_k = 2\left(\frac{1}{2}\right)^k u_k -3\left(\frac{1}{3}\right)^k u_k$ \\
Pour avoir un signal anticausal, on prend $|z| < \frac{1}{3}$ et ainsi, $x_k = -2\left(\frac{1}{2}\right)^k u_{-k-1} +3 \left(\frac{1}{3}\right)^k u_{-k-1}$ \\
Pour avoir un signal bilatéral, on prend $\frac{1}{3}<|z| < \frac{1}{2}$ et ainsi, $x_k = -2\left(\frac{1}{2}\right)^k u_{-k-1} -3\left(\frac{1}{3}\right)^k u_k$ \\
Ainsi, on voit que le nombre de pôle entraîne le nombre de régions de convergence.

\subsection{Transformée de Fourier des signaux à temps discret}
Pour les signaux discret, on définit la transformée de Fourier par : \[(f)=\sum_{k=-\infty}^{+\infty} x_k e^{-2\pi jkf}\]
X(f) est une fonction péridoique de fréquence 1. On étudie donc X(f) sur $[-\frac{1}{2}, \frac{1}{2}]$ ou sur $[0,1]$
On remarque que si le cercle unité appartient à la région de convergence de la tranformée en z, $X(f)={X(z)}|_{z=e^{2\pi jf}}$
