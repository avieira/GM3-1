\documentclass{article}

\usepackage[utf8x]{inputenc}
\usepackage[T1]{fontenc}
\usepackage[francais]{babel}
\usepackage{lmodern}
\usepackage{amsthm}
\usepackage{amsmath}
\usepackage{amssymb}
\usepackage{mathrsfs}
\usepackage{verbatim}
\usepackage{moreverb}
\usepackage[top=3cm, bottom=2cm, left=3cm, right=3cm]{geometry}
\usepackage{listings}
\usepackage{graphicx}
\usepackage{hyperref}

\newtheoremstyle{mes_theoremes}{}{}{}{}{\bfseries}{~:\newline}{\parindent}{\thmname{#1}\thmnumber{ #2}\thmnote{ (#3)}}
\theoremstyle{mes_theoremes}

\newtheorem*{Def}{Définition}
\newtheorem*{Prop}{Proposition}
\newtheorem*{dem}{Demonstration}
\newtheorem*{theo}{Théorème}
\newtheorem*{rmq}{Remarque}
\newtheorem*{lem}{Lemme}
\newtheorem*{coro}{Corollaire}

\newcommand{\Inde}{\perp \! \! \! \perp}

\hypersetup{colorlinks=true, linkcolor=red}

\begin{document}

\setcounter{tocdepth}{4}
\tableofcontents
\newpage

\part{Probabilités}
\section{Espaces probabilisés dénombrables}
Soit $\Omega = (\omega_n)_{n \geqslant 1} $ un ensemble dénombrable, et soit $f: \Omega \rightarrow \mathbb{R}^{+} $ une fonction.

\[\lim_{n \rightarrow +\infty}{\sum_{\omega \in F} f(\omega)} = \sup \{ \sum_{\omega \in F} f(\omega) ; F \subset \Omega\ et\ F\ fini \} = \sum_{\omega \in \Omega} f(\omega) \]

\begin{Def}
On appelle probablité sur \begin{math} \Omega \end{math} une fonction P définie sur \begin{math}\mathcal{P}(\Omega)\end{math} et à valeur dans [0,1] tel que :
\begin{enumerate}
\item $P(\Omega)=1$
\item Pour toute famille dénombrable $\mathcal{A}$ de partie de $ \Omega $ 2 à 2 incompatibles, on a : \[P(\bigcup_{A \in \mathcal{A}} A)=\sum_{A \in \mathcal{A}} P(A) \]
\end{enumerate}
\end{Def}

\begin{Prop}
Soit $\mathbb{P}$ une fonction définie sur $\mathcal{P}(\Omega)$ à valeur dans $\mathbb{R}^{+}$. $\mathbb{P}$ vérifie la $\sigma$-additivité si et seulement si :
\begin{enumerate}
\item $P(A \cup B) = P(A) + P(B)$ pour tout A et B incompatibles
\item Pour toute suite $(A_n)_{n \geqslant 1}$ croissante de parties de $\Omega$, \[\mathbb{P}(\bigcup_{n=1}^{+\infty} A_n) = \lim_{n \rightarrow +\infty} \mathbb{P}(A_n)\]
\end{enumerate}
\end{Prop}

\begin{dem}
$\sigma \Rightarrow$ 1 : évident
$\sigma \Rightarrow$ 2 : Soit $(A_n)_{n \geqslant 1}$ croissante. Posons $B_{k+1}=A_{k+1} \textbackslash{} A_{k}$
$\bigcup_{n \geqslant 1} A_n = \bigcup_{n \geqslant 1} B_n$ et les $(A_n)_{n \geqslant 1}$ sont 2 à 2 incompatibles.
D'après la propriété de $\sigma$-additivité, 
\begin{eqnarray*}
P(\bigcup_{n \geqslant 1} A_n) &=& P(\bigcup_{n \geqslant 1} B_n) \\
&=& \lim_{n \rightarrow +\infty} \sum_{p=0}^{n} P(B_n) \\
&=& \lim_{n \rightarrow +\infty} P(A_0) + P(A_n) - P(A_0) \\
&=& \lim_{n \rightarrow +\infty} P(A_n)
\end{eqnarray*}

1 et 2 $\Rightarrow \sigma$ : Soit $(A_n)_{n \geqslant 1}$ 2 à 2 incomptaibles. Posons $B_n = \bigcup_{k=1}^n A_k$ et $\bigcup_{n \geqslant 1} A_n = \bigcup_{n \geqslant 1} B_n$.
\begin{eqnarray*}
P(\bigcup_{n \geqslant 1} A_n) &=& P(\bigcup_{n \geqslant 1} B_n) = \lim_{n \rightarrow +\infty} P(B_n) = \lim_{n \rightarrow +\infty} P(\bigcup_{k=1}^n A_k) \\
&=& \lim_{n \rightarrow +\infty} \sum_{k=1}^n P(A_k) = \sum_{k=1}^{+\infty} P(A_k)
\end{eqnarray*}
\end{dem}

\begin{Def}
On appelle espace dénombrable probabilisé un couple $(\Omega, \mathbb{P})$ où $\Omega$ est un ensemble dénombrable non vide et $\mathbb{P}$ une mesure de probabilité sur $\Omega$.
\end{Def}

\begin{Prop}
Les proproétés 1 à 7 vues dans le cas fini restent vraies. De plus, pour toute famille $\mathcal{A}$ dénombrable d'évènement, on a 
\[\mathbb{P}(\bigcup_{A \in \mathcal{A}} A) \leqslant \sum_{A \in \mathcal{A}} \mathbb{P}(A)\]
\end{Prop}

Remarque : La notion de densité discrète se généralise au cas où $\Omega$ est infini dénombrable.

\addcontentsline{toc}{subsubsection}{Lois usuelles sur $\mathbb{N}$} 
\subsection*{Lois usuelles sur $\mathbb{N}$}
\paragraph{Loi géométrique}
Pour tout $p\in]0,1[$, la fonction f définie pour tout $k\in\mathbb{N}^*$ par \[f(k)=(1-p)^{k-1}p\] est une densité de probabilité sur $\mathbb{N}^*$
On appelle la probabilité sur $\mathbb{N}^*$ associée, la loi géométrique de paramètre p et on la note $\mathfrak{G}(p)$. 

\paragraph{Loi binomiale négative}
Pour tout $p\in]0,1[$ et $r\in\mathbb{N}^*$, la fonction f définie pour tout $k\in\mathbb{N}^*$ par \[f(k)=\left(\begin{array}{r}k-1 \\ r-1 \end{array}\right)(1-p)^{k-r}p^r\] est une densité de probabilité sur $\mathbb{N}^*$
On appelle la probabilité sur $\mathbb{N}^*$ associée, la loi binomiale négative de paramètre r et p et on la note $\mathfrak{BN}(r,p)$. 

Remarque : $\mathfrak{BN}(1,p) = \mathfrak{G}(p)$ et f(k)=0 si r>k.
Ici, on ne s'intéresse pas au 1\ier\ succès mais au r-ième succès.

\paragraph{Loi de Poisson}
Pour tout $\lambda$>0, la fonction f définie pour tout $k\in\mathbb{N}$ par \[f(k)=e^{-\lambda}\frac{\lambda^k}{k!}\] est une densité de probabilité sur $\mathbb{N}$
On appelle la probabilité sur $\mathbb{N}$ associée, la loi géométrique de paramètre $\lambda$ et on la note $\mathfrak{P}(\lambda)$. 

\begin{theo}
Soit $(p_n)_{n \geqslant 1}$ une suite d'éléments de [0,1] tel que \[\lim_{n\rightarrow+\infty} np_n =\lambda\]
Alors \[\lim_{n\rightarrow+\infty} \left(\begin{array}{r}n \\ k \end{array}\right) p_n^k(1-p_n)^{n-k} = e^{-\lambda}\frac{\lambda^k}{k!}\]
\end{theo}

\begin{dem}
$np_n = \lambda u_n$ avec $\lim_{n\rightarrow+\infty} u_n = 1$.
Soit $k \in \mathbb{N}$ fixé.
\[\left(\begin{array}{r}n \\ k \end{array}\right) p_n^k(1-p_n)^{n-k} = \frac{n!}{k!(n-k)!} \times \frac{1}{n^k} (\lambda u_n)^k \left(1-\lambda \frac{u_n}{n}\right)^{n-k}\]
Or, $\left(1-\frac{x}{n}\right)^n \rightarrow e^{x}$ et $\frac{n!}{(n-k)!} \frac{1}{n^k} \rightarrow 1$ d'où le résultat.
\end{dem}

En pratique, on conviendra que l'approximation d'une binomiale par une loi de Poisson de paramètre np est correcte pour $n \geqslant 50$, $n \leqslant 0,01$ et $np \leqslant 10$. 


\section{Espaces probabilisés généraux}
\begin{Def}
Soit $\Omega$ un ensemble quelconque non vide.
Une tribu $\mathfrak{F}$ sur $\Omega$ est un ensemble de parties de $\Omega$ vérifiant :
\begin{enumerate}
\item $\Omega \in \mathfrak{F}$
\item $\mathfrak{F}$ stable par complémentaire
\item $\mathfrak{F}$ stable par union dénombrable
\end{enumerate}
Le couple $(\Omega, \mathfrak{F})$ est appelé espace probabilisable. Un élément de $\mathfrak{F}$ est appelé évenement.
\end{Def}

\begin{Def}
On appelle tribu borélienne de $\mathbb{R}^d$ et on note $\mathfrak{B}(\mathbb{R}^d)$ la tribu engendrée par l'ensemble des ouverts de $\mathbb{R}^d$, c'est-à-dire la plus petite tribu qui contient l'ensemble des ouverts de $\mathbb{R}^d$.
\end{Def}

\begin{Def}
On appelle mesure de probabilité ou loi de probabilité sur l'espace mesurable (ou probabilisable) $(\Omega, F)$ une application définie sur $\mathcal{F}$ et à valurs dans $\mathbb{R}^+$ tel que :
\begin{enumerate}
\item $\mathbb{P}(A) \in [0,1] \forall A \in \mathcal{F}$
\item $\mathbb{P}(\Omega)=1$
\item Pour toute famille $\mathcal{A}$ dénombrable d'éléments deux à deux incompatibles de $\mathcal{F}$, on a \[\mathbb{P}(\bigcup_{A \in \mathcal{A}} A) = \sum_{A \in \mathcal{A}} \mathbb{P}(A)\]
\end{enumerate}
\end{Def}

\begin{Def}
On appelle densité de probabilité (continue) sur $\mathbb{R}$ (ou sur $\mathbb{R}^d$) une fonction f continue par morceaux à valeurs positives et tel que \[\int_{\mathbb{R}}f(x)dx=1\] (ou $\int_{\mathbb{R}^d}f(x)dx=1$).
\end{Def}

\begin{theo}
Si f est une densité de probabilité sur $\mathbb{R}$ (ou $\mathbb{R}^d$) alors il existe une unique mesure de probabilité $\mathbb{P}$ sur ($\mathbb{R}$, $\mathfrak{B}(\mathbb{R})$) (ou sur ($\mathbb{R}^d$, $\mathfrak{B}(\mathbb{R}^d)$) tel que pour tout intervalle ]a,b[ (ou tout cylindre $]a_1,b_1[\times...\times]a_n,b_n[$) on ait : 
\[\mathbb{P}(]a,b[)=\int_a^b f(x)dx\ ou \  \mathbb{P}(]a_1,b_1[\times...\times]a_n,b_n[)=\int_{a_1}^{b_1}...\int_{a_n}^{b_n} f(x)dx\]
\end{theo}

\addcontentsline{toc}{subsubsection}{Exemples de lois continues}
\subsection*{Exemples de lois continues}
\paragraph{Loi normale $\mathcal{N}$(0,1)}
X$\hookrightarrow \mathcal{N}(0,1)$ \\
\begin{eqnarray*}
f_X(x)&=&\frac{1}{\sqrt{2\pi}} e^{\frac{-x^2}{2}} \\
F_X(x)&=&\int_{-\infty}^{+\infty} f_X(t)dt \\
\mathbb{P}(a \leq X \leq b) &=& \int_{a}^{b} f_X(t)dt
\end{eqnarray*}

\paragraph{Loi normale $\mathcal{N}(\mu,\sigma^2)$}
X$\hookrightarrow \mathcal{N}(\mu,\sigma^2)$ \\
\[f_X(x)=\frac{1}{\sqrt{2\pi}\sigma} e^{\frac{-(x-\mu)^2}{2\sigma^2}}\]

\paragraph{Loi exponentielle $\mathcal{E}(\lambda)$}
X$\hookrightarrow \mathcal{E}(\lambda)$ \\
\[f_X(x)=\lambda e^{-\lambda x}\ 1_{\mathbb{R}_+^*}(x)\]

\paragraph{Loi uniforme sur l'intervalle ]a,b[}
X$\hookrightarrow \mathcal{U}(]a,b[)$ \\
\[f_X(x)=\frac{1}{b-a} \ 1_{]a,b[}(x)\]

\paragraph{Loi uniforme sur un borélien $\mathfrak{B}$ de $\mathbb{R}^2$}
X$\hookrightarrow \mathcal{U}(B)$ avec $B \in \mathfrak{B}(\mathbb{R}^2)$ fixé\\
$\forall A \in \mathfrak{B}(\mathbb{R}^2)$, $\mathbb{P}(X\in A)=\frac{\lambda_2 (A \cap B)}{\lambda_2(B)}$
\[f_X(x)=\frac{1}{\lambda_2(B)} \ 1_{B}(x)\]

Autres lois : Gamma, Bêta, de Student, du $\chi^2$, de Fisher...

\begin{Def}
On appelle v.a. réelle (ou d-dimensionnelle) à valeur dans un borélien E une application mesurable X définie sur $\Omega$ à valeurs dans E, ie \[\forall A \in \mathfrak{B}(E)\{x \in A\}=X^{-1}(A) \in \mathcal{F}\]
Si E=$\mathbb{R}$, on dit que X est une v.a. réelle \\
Si E=$\mathbb{R}^d$, on dit que X est un vecteur aléatoire réel
\end{Def}

\begin{Prop}
\begin{enumerate}
\item La somme et le produit d'un nombre fini de v.a.r. est une v.a.r.
\item Si $(X_n)_{n \geq 1}$ est une suite de v.a.r. tel que $X_n(\omega) \xrightarrow[n \rightarrow +\infty]{} X(\omega)$ pour tout $\omega \in \Omega$ alors l'application X est une v.a.r
\end{enumerate}
\end{Prop}

\begin{Def}
Soit X une v.a.r. à valeur dans $E\in \mathfrak{B}(\mathbb{R})$. L'application Q définie pour tout $B\in \mathfrak{B}(\mathbb{R})$ par \[Q(B)=\mathbb{P}(X\in B)\] est une mesure de probabilité. On l'appelle loi de probabilité sur E de la v.a. X.
\end{Def}

\paragraph{Loi exponentielle}\
Soit $\lambda$>0. On dit qu'une var X suit la loi exponentielle de paramètre $\lambda$>0 si \[\forall 0<a<b, \mathbb{P}(X\in[a,b])=\int_a^b \lambda e^{-\lambda x}\]
On note alors $X \hookrightarrow \mathcal{E}(\lambda)$

\paragraph{Remarque} : $\mathbb{P}(X\in ]0,+\infty[)=1$ et donc nécessairement, $\mathbb{P}(X\leq 0)=0$.

\begin{Prop}
Si $X\hookrightarrow \mathcal{E}(\lambda)$ avec $\lambda$ >0, alors $aX \hookrightarrow \mathcal{E}(\frac{\lambda}{a})$, $\forall a>0$.
En particulier, $\lambda X \hookrightarrow \mathcal{E}(1)$
\end{Prop}

\begin{dem}
La fonction de répartition de $\mathcal{E}(\lambda)$ est \[F_X(x) = \left( 1-e^{-\lambda x} \right) 1_{\mathbb{R}_+^*} (x)\]
Posons Y=aX. \\
\begin{eqnarray*}
F_Y(x) &=& \mathbb{P}(Y \leq x) \\
&=& \mathbb{P} \left( X \leq \frac{x}{a} \right) \\
&=& F_X \left( \frac{x}{a} \right) \\
&=& \left( 1-e^{-\frac{\lambda}{a} x} \right) 1_{\mathbb{R}_+^*} (x)
\end{eqnarray*}
Il s'agit de la fonction de répartition de la loi $\mathcal{E}(\frac{\lambda}{a})$
\end{dem}

\begin{Prop}
Pour rappel, [X]=E(X) (partie entière de X). \\
Si $X\hookrightarrow \mathcal{E}(\theta)$, alors $[X]+1 \hookrightarrow \mathcal{G}(1-e^{-\theta})$
\end{Prop}

\begin{dem}
On pose Z=[X]+1. On a $Z(\Omega)=\mathbb{N}^*$. \\
Soit $k\in \mathbb{N}^*$.
\begin{eqnarray*}
\mathbb{P}(Z=k) &=& \mathbb{P}([X]=k-1) \\
&=& \mathbb{P}(k-1 \leq X < k) \\
&=& \int_{k-1}^k \theta e^{-\theta x} dx \\
&=& \left[ -e^{-\theta x} \right]_{k-1}^k \\
&=& e^{-(k-1) \theta} - e^{-k \theta} \\
&=& e^{-\theta (k-1)} \left( 1-e^{- \theta} \right)
\end{eqnarray*}
En posant $p=1-e^{-\theta}$, $\mathbb{P}(Z=k) = (1-p)^{k-1} p$. \\
Donc $[X]+1 \hookrightarrow \mathcal{G}(1-e^{-\theta})$.
\end{dem}

\begin{Prop}
De la même manière, on peut montrer que : \\
Soit $X \hookrightarrow \mathcal{E}(\theta)$ et $(N_n)_{n\geq 0} \in \left( \mathbb{R}_+^* \right)^{\mathbb{N}}$ tel que $N_n \xrightarrow[n \rightarrow +\infty]{} +\infty$.
\begin{enumerate}
\item Pour tout $n \geq 0$, $[N_n X] \hookrightarrow \mathcal{G}\left( 1-e^{-\frac{\theta}{N_n}} \right)$ 
\item $\lim_{n \rightarrow +\infty} \frac{[N_n X]}{N_n} = X$ de façon presque sûr (ie $\mathbb{P} \left( \left\{ \omega \in \Omega | \lim_{n\rightarrow +\infty} \frac{[N_n X](\omega)}{N_n}=X(\omega) \right\} \right) =1$)
\end{enumerate}
\end{Prop}

\begin{theo}
Soit X une v.a. à valeurs dans $\mathbb{R}^+$. Les propriétés suivantes sont équivalentes :
\begin{enumerate}
\item X est une v.a. de loi exponentielle
\item Pour tout réel t>0, $\mathbb{P}(X>t) \neq 0$ et \[\forall s>0, \mathbb{P}(X>t+s | X>t) = P(X>s)\] (Propriété sans mémoire)
\end{enumerate}
\end{theo}

\begin{dem}[de la première implication]
t>0 et s >0 \\
\begin{eqnarray*}
\mathbb{P}(X>t+S | X>t) &=& \frac{\mathbb{P}\left( (X>t+s) \cap (X>t) \right)}{\mathbb{P}(X>t)} = \frac{\mathbb{P}(X>t+s)}{\mathbb{P}(X>t)} \\
&=& \frac{1-F_X(s+t)}{1-F_X(x)} \\
&=& \frac{e^{-\theta (s+t)}}{e^-{\theta t}}=e^{-\theta s} \\
&=& 1-F_X(s) \\
&=& \mathbb{P}(X>s)
\end{eqnarray*}
\end{dem}

\begin{theo}
Soit X une v.a. à valeurs dans $\mathbb{N}^*$. Il t a équivalence entre :
\begin{enumerate}
\item X est une v.a. de loi géométrique
\item $\forall n\in \mathbb{N}$, $\mathbb{P}(X>n) \neq 0$, et : \[\forall k\in \mathbb{N}, \mathbb{P}(X>n+k | X>n) = \mathbb{P}(X>k)\]
\end{enumerate}
\end{theo}

\section{Notion d'indépendance}
Soit $(\Omega, \mathcal{F}, \mathbb{P})$ un espace probabilisé.

\begin{Def}\
\begin{enumerate}
\item On dit que deux évenements A et B sont indépendants s'ils vérifient $\mathbb{P}(A \cap B) = \mathbb{P}(A) \mathbb{P}(B)$.
\item Soit $(A_i)_{i\in I}$ une famille quelconque d'évenements. On dit que les $(A_i)_{i\in I}$ sont mutuellement indépendants si pour tout ensemble fini d'indices distincts $\{i_1,...,i_n\} \subset I$, on a \[\mathbb{P}(A_{i_1}\cap ... \cap A_{i_k})=\mathbb{P}(A_{i_1}) ... \mathbb{P}(A_{i_k})\]
\end{enumerate}
\end{Def}

Soient $(\Omega_1, \mathcal{F}_1, \mathbb{P}_1)$ et $(\Omega_2, \mathcal{F}_2, \mathbb{P}_2)$ deux espaces probabilisés. Considérons l'espace probabilisable $(\Omega, \mathcal{F})$ où $\Omega = \Omega_1 \times \Omega_2$ et $\mathcal{F}=\mathcal{F}_1 \otimes \mathcal{F}_2$ (produit tensoriel) est la tribu engendrée par les parties de $\Omega$ de la forme $A_1 \times A_2$ avec $A_1 \in \mathcal{F}_1$ et $A_2 \in \mathcal{F}_2$.

\begin{Prop}
$\exists! \mathbb{P}$ sur $(\Omega,\mathcal{F})=(\Omega_1 \times \Omega_2, \mathcal{F}_1 \otimes \mathcal{F}_2)$ tel que : 
\[\forall A_1 \in \mathcal{F}_1, \forall A_2 \in \mathcal{F}_2, \mathbb{P}(A_1 \times A_2)=\mathbb{P}_1(A_1) \mathbb{P}_2(A_2)\]
$\mathbb{P}$ s'appelle le produit tensoriel de $\mathbb{P}_1$ et $\mathbb{P}_2$ et on note $\mathbb{P}=\mathbb{P}_1 \otimes \mathbb{P}_2$.
\end{Prop}

Soient $(\Omega, \mathcal{A}, \mathbb{P})$ un espace probabilisé. On note X et Y deux variables aléatoires définies sur $(\Omega, \mathcal{A}, \mathbb{P})$ tel que X soit à valeur dans un espace mesurable (E,$\mathcal{E}$) et Y à valeur dans un espace mesurable (F,$\mathcal{F}$).

\begin{Def}
On dit que X et Y sont indépendantes (et on note X$\Inde$Y) si pour tout $A\in\mathcal{E}$ et tout $B\in\mathcal{F}$ les évenements $\{X\in A\}$ et $\{Y \in B\}$ sont indépendants.\\
De même, soit $(X_i)_{i\in I}$ une famille de v.a. dans les espaces mesurables $((E_i,\mathcal{E}_i))_{i\in I}$. On dit que les $(X_i)_{i\in I}$ sont indépendantes entre elles si pour toute famille $(B_i)_{i \in I}$ d'éléments de $(\mathcal{E}_i)_{i\in I}$, les éléments $\{X_i \in B_i\}_{i\in I}$ sont mutuellement indépendants.
\end{Def}

\begin{Prop} \
\begin{enumerate}
\item X$\Inde$Y $\Rightarrow$ f(X)$\Inde$g(Y) pour tout f et g mesurables
\item Si la loi du couple (X,Y) admet une densité, alors \[X\Inde Y \Leftrightarrow f_{(X,Y)}(x,y)=f_X (x) f_Y (y)\]
\end{enumerate}
\end{Prop}

\section{Fonction de répartition}
\subsection{Définition générale}
\begin{Def} \
\begin{itemize}
\item Pour toute mesure de probabilité $\mu$ sur $(\mathbb{R},B_{\mathbb{R}})$, on apelle fonction de répartition de $\mu$, notée $F_{\mu}$, et définie pour tout $x\in \mathbb{R}$ par : \[F_{\mu}(x)=\mu(]-\infty, x])\]
\item Soit X une v.a. réelle définie sur $(\Omega, \mathcal{F}, \mathbb{P})$. On parlera alors de fonction de répartition de la v.a. X au lieu de la fonction de répartition de la loi $\mu_X$ de X. On notera $F_{\mu} = F_X$ et on aura donc : \[F_X(x) = \mu_X (]-\infty,x])=\mathbb{P}(X \leq x)\]
\item Plus généralement, si X=$(X_1,...,X_d)$ est un vecteur aléatoire de $\mathbb{R}$, alors la fonction de répartition $F_X$ de X est définie pour tout x=$(x_1,...,x_d)\in \mathbb{R}^d$ par \[F_X(x)=\mathbb{P}(X \leq x) \mathbb{P}(X_1<x_1,...,X_d<x_d)\]
\end{itemize}
\end{Def}

\begin{theo}[admis]
Deux mesures de probabilité sur $(\mathbb{R}, B_{\mathbb{R}})$ sont égales ssi elles ont même fonction de répartition.
\end{theo}

\begin{Prop}[de la f.d.r]
Soit X une v.a. définie sur $(\Omega,\mathcal{F},\mathbb{P})$ et soit $F_X$ sa f.d.r.
\begin{itemize}
\item $F_X$ est croissante tel que $\lim_{x \to +\infty} F_X(x)=1$ et $\lim_{x \to -\infty} F_X(x)=0$
\item $F_X$ est "Cadlag" (continue à droite et limité à gauche) et pour tout $\alpha\in\mathbb{R}$ on a : \[F_x(\alpha^{-})=\lim_{\underset{x<\alpha}{x \to \alpha}} F_X(x) = \mathbb{P}(X<\alpha)\]
\item \[\forall \alpha \in \mathbb{R}, \mathbb{P}(X=\alpha)=F_X(\alpha) - F_X (\alpha^{-})\]
\end{itemize}
\end{Prop}

\begin{dem}[de la dernière propriété]
\begin{eqnarray*}
F_X(\alpha)-F_X(\alpha^-)&=&\lim_{n\to +\infty} \mathbb{P}(X\leq \alpha) -\mathbb{P}(X \leq \alpha - \frac{1}{n}) \\
&=& \lim_{n\to +\infty} \mathbb{P}(\alpha - \frac{1}{n} < X\leq \alpha) \\
&=& \lim_{n\to +\infty} \mathbb{P}(X \in ]\alpha-\frac{1}{n}, \alpha]) \\
&=& \mathbb{P}(X \in \bigcap_n \alpha-\frac{1}{n}, \alpha]) \\
&=& \mathbb{P}(X = a)
\end{eqnarray*}
\end{dem}

\begin{Prop}
Soit X une variable aléatoire réelle de densité $f_X$. La f.d.r. $F_X$ de X vérifie :
\begin{itemize}
\item $\forall x\in \mathbb{R}, F_X(x)=\int_{-\infty}^x f_X(t)dt$
\item $F_X$ continue sur $\mathbb{R}$
\item Si $f_X$ est continue en $x_0 \in \mathbb{R}$ alors $F_X$ est dérivable et $F'(x_0)=f(x_0)$
\end{itemize}
\end{Prop}

\begin{Prop}
On suppose que la f.d.r. $F_X$ de la var X est $\mathcal{C}^1$ par morceaux au sens suivant :
\begin{itemize}
\item $F_X$ continue sur $\mathbb{R}$ sauf éventuellement en un nombre fini de points $a_1<a_2<...<a_n$.
\item Sur chacun des des intervalles $]-\infty,a_1[,]a_n,+\infty[$ et $]a_i, a_{i+1}[$ pour tout $1 \leq i \leq n-1$, la dérivée f de $F_X$ est continue.
\end{itemize}
Alors X a pour densité $f$
\end{Prop}

\subsection{Espérance et variance d'une var}
\begin{Def}
Soit X une var définie sur l'espace probabilisé $(\Omega, \mathcal{F}, \mathbb{P})$. On appelle espérance de la v.a. X et on note E(X) l'intégrale (au sens de Lebesgue) \[\int_{\Omega} X(\omega) d\mathbb{P}(\omega)\] lorsque celle-ci est bien définie. \\
Si E(|X|) est un nombre fini, on dit que la v.a. X est intégrable.
\end{Def}

\begin{rmq}
Posons $X^+=\sup(X,0)$ et $X^-=\inf((-X),0)$. ($X^+ \geq 0$ et $X^- \geq 0$ p.s.) \\
Les intégrales $\int_{\Omega} X^+ d\mathbb{P}$ et $\int_{\Omega} X^- d\mathbb{P}$ ont toujours un sens.
Comme $X=X^+ + X^-$ on pose \[\int_{\Omega} X d\mathbb{P} = \int_{\Omega} X^+ d\mathbb{P} + \int_{\Omega} X^- d\mathbb{P}\]
\end{rmq}

\begin{theo}
\begin{enumerate}
\item Si $X(\Omega)$ dénombrable, alors : \[E(X) = \sum_{k\in X(\Omega)} k \mathbb{P}(X=k)\] et pour toute fonction g définie sur $X(\Omega)$ à valeur dans $\mathbb{R}$ : 
\[E(g(X))=\sum_{k\in X(\Omega)} g(k) \mathbb{P}(X=k)\]
\item Si la loi de X admet une densité $f_X$ alors \[E(X)=\int_{\mathbb{R}} xf_X(x)dx\] et plus généralement, si g est une fonctions mesurable sur $\mathbb{R}$ à valeur dans $\mathbb{R}$ alors : \[E(g(X))=\int_{\mathbb{R}} g(x) f_X(x) dx\]
\end{enumerate}
\end{theo}

\begin{Prop}
Soient X et Y deux v.a.r.
\begin{enumerate}
\item $\forall a,b\in \mathbb{R}, E(aX+bY)=aE(X)+bE(Y)$
\item Si $X\geq 0$ p.s. alors $E(X) \geq 0$
\item Si $X \geq Y$ p.s. alors $E(X) \geq E(Y)$
\item $|E(X)| \leq E(|X|)$ et plus généralement, si $\phi$ est une fonction convexe, alors \[\phi(E(X)) \leq E(\phi(X))\] (inégalité de Jensen)
\end{enumerate}
\end{Prop}

\begin{theo}[de la convergence dominée de Lebesgue]
Soit $(X_n)_{n\geq 1}$ une suite de v.a.r. qui converge p.s. vers une v.a.r. X.\\
S'il existe une v.a. Y intégrable tel que $|X_n| \leq Y$ p.s. $\forall n\geq 1$ alors \[E(X) = \lim_{n \to +\infty} E(X_n)\]
\end{theo}

\begin{Prop}[Inégalité de Markov]
Soit X est une v.a.r. définie sur $(\Omega,\mathcal{F},\mathbb{P})$ \\
Si $X \geq 0$ p.s. alors \[\forall \lambda > 0,  \mathbb{P}(X > \lambda) \leq \frac{E(X)}{\lambda}\]
\end{Prop}

\begin{dem}
\[\forall A\in \mathcal{F},\ \mathbb{P}(A)=\int_A d\mathbb{P}=\int 1_A d\mathbb{P}=E(1_A)\]
\begin{eqnarray*}
\mathbb{P}(X \leq \lambda) &=& E(1_{\{X \leq \lambda\}}) = \int_{\Omega} 1_{\{X \leq \lambda\}} d\mathbb{P} \\
&\leq& \int_{\Omega} \frac{X}{\lambda} 1_{\{X \leq \lambda\}} d\mathbb{P}\\
&\leq& \int_{\Omega} \frac{X}{\lambda} d\mathbb{P} (car\ X\geq 0\ p.s.) \\
&\leq& \frac{1}{\lambda} \int_{\Omega} X d\mathbb{P} \\
&\leq& \frac{E(X)}{\lambda}
\end{eqnarray*}
\end{dem}

\begin{Def}
Pour toute v.a.r. X, on appelle variance de X le nombre (s'il existe) \[V(X)=E\left((X-E(X))^2 \right) = E\left(X^2\right) - \left( E(X)\right)^2\]
\end{Def}

\begin{Prop}[Inégalité de Bienaymé-Tcheychev]
\[\forall \lambda > 0,  \mathbb{P}(|X-E(X)| > \lambda) \leq \frac{V(X)}{\lambda^2}\]
\end{Prop}

\begin{dem}
\begin{eqnarray*}
\mathbb{P}(|X-E(X)| > \lambda) &=& \mathbb{P}\left((X-E(X))^2 > \lambda^2 \right) \\
&\leq& \frac{E\left( (X-E(X))^2 \right)}{\lambda^2} \\
&\leq& \frac{V(X)}{\lambda^2}
\end{eqnarray*}
\end{dem}

\begin{Prop}
\begin{itemize}
\item $V(X)\geq 0$ (car $E(X)^2 \leq E(X^2)$ )
\item Si la loi de X dmet une densité $d_x$ alors \[V(X)=\int_{\mathbb{R}} x^2f_X(x) dx - \left(\int_{\mathbb{R}} xf_X(x) dx\right)^2\]
\item $\forall(a,b)\in\mathbb{R}^2,\ V(aX+b)=a^2V(X)$
\item Considérons la fonction \begin{eqnarray*} g:\mathbb{R}&\rightarrow&\mathbb{R}^+ \\ a&\mapsto& E((X-a)^2) \end{eqnarray*} alors \[\underset{a\in\mathbb{R}}{\mathrm{argmin}}\ g(a)=E(X)\ et\ \min_{a\in\mathbb{R}} g(a)=V(X)\]
\item $X\Inde Y \Rightarrow V(X+Y)=V(X)+V(Y)$
\end{itemize}
\end{Prop}

\paragraph{Exemples à connaître : \\}
\begin{itemize}
\item $X\hookrightarrow \mathcal{B}(p)$ alors E(X)=p et V(X)=p(1-p)
\item $X\hookrightarrow \mathcal{B}(p,n)$ alors E(X)=np et V(X)=np(1-p)
\item $X\hookrightarrow \mathcal{G}(p)$ alors E(X)=$\frac{1}{p}$ et V(X)=$\frac{1-p}{p^2}$
\item $X\hookrightarrow \mathcal{P}(\lambda)$ alors E(X)=V(X)=$\lambda$
\item $X\hookrightarrow \mathcal{U}(]a,b[)$ alors E(X)=$\frac{a+b}{2}$ et V(X)=$\frac{(b-a)^2}{12}$
\item $X\hookrightarrow\mathcal{E}(\theta)$ alors E(X)=$\frac{1}{\theta}$ et V(X)=$\frac{1}{\theta^2}$
\item $X\hookrightarrow \mathcal{N}(\mu,\sigma^2)$ alors E(X)=$\mu$ et V(X)=$\sigma^2$
\end{itemize}

\section{Convergence d'une variable aléatoire}
\subsection{Convergence en probabilité et presque sûr}
\begin{Def}
Soit $(Y_n)_{n\geq 1}$ une suite de v.a.r. et soit Y une v.a.r.
\begin{enumerate}
\item On dit que $(Y_n)_{n\geq1}$ converge en probabilité vers Y si : \[\forall \varepsilon>0, \mathbb{P}(|Y_n-Y|\geq \epsilon) \xrightarrow[n \to +\infty]{} 0\]
On note \[Y_n\xrightarrow[n \to +\infty]{\mathbb{P}}Y\]
\item On dit que $(Y_n)_{n\geq1}$ converge presque-sûrement vers Y si : \[\mathbb{P}(\{\omega\in\Omega|\lim_{n\to+\infty}Y_n(\omega)=Y(\omega)\})=1\]
On note \[Y_n\xrightarrow[n \to +\infty]{p.s.}Y\]
\end{enumerate}
\end{Def}

\begin{Prop}
La convergence p.s. entraîne la convergence en probabilité.
\end{Prop}

\begin{dem}
A reprendre
\end{dem}

\begin{Prop}
Soient Y et $(Y_n)_{n\geq 1}$ des v.a.r. telles que \[\forall \varepsilon>0, \sum_{n\geq 1} \mathbb{P}(|Y_n-Y|>\varepsilon)<+\infty\]
alors \[Y_n \xrightarrow[n\to +\infty]{p.s.} Y\]
\end{Prop}

\begin{dem}
Posons $B_{n,\varepsilon}=\{|Y_n-Y|>\varepsilon\}$ et $A_{\varepsilon}=\overline{\lim}_{n\to +\infty}B_{n,\varepsilon}$ \\
D'après le lemme de Borel-Cantelli : \[\mathbb{P}(A_{\varepsilon})=0\ \forall\varepsilon>0\]
Or, $A_{\varepsilon}=\bigcap_{k\geq 1} \bigcup_{k\geq n} B_{k,\varepsilon}$ et $\overline{A_{\varepsilon}}=\bigcup_{k\geq 1} \bigcap_{k\geq n} \overline{B_{k,\varepsilon}}$ \\
On a $\mathbb{P}(\overline{A_{\varepsilon}})=1$ $\forall \varepsilon>0$. Posons $E=\bigcap_{s\in\mathbb{N}^*} \overline{A_{\frac{1}{s}}}$ 
\[\mathbb{P}(\overline{E})=\mathbb{P}(\bigcap_{s\in\mathbb{N}^*} A_{\frac{1}{s}}) \leq \sum_{s\in\mathbb{N}^*} \mathbb{P}(A_{\frac{1}{s}})=0\]
D'où $\mathbb{P}(E)=1$
\begin{eqnarray*}
\omega\in E &\Leftrightarrow& \forall s\in \mathbb{N}^*, \omega \in \overline{A_{\frac{1}{s}}} \\
&\Leftrightarrow& \forall s\in \mathbb{N}^*, \exists n>1, \forall k\geq n, \omega \in B_{k,\frac{1}{s}} \\
&\Leftrightarrow& \forall s\in \mathbb{N}^*, \exists n>1, \forall k\geq n, |Y_k(\omega)-Y(\omega)|\leq \frac{1}{s} \\
&\Leftrightarrow& \forall \varepsilon>0, \exists n>1, \forall k\geq n, |Y_k(\omega)-Y(\omega)|\leq \varepsilon \\
&\Leftrightarrow& Y_k \xrightarrow[k\to +\infty]{} Y
\end{eqnarray*}
\end{dem}

\subsection{Covariance de deux variables aléatoires réelles}
\begin{Def}
Soient X et Y deux v.a.r. \\
La covariance de X et Y, notée cov(X,Y) est définie par : \begin{eqnarray*}
\text{cov(X,Y)}&=&E((X-E(X)(Y-E(Y)) \\
&=& E(XY)-E(X)E(Y)
\end{eqnarray*}
Si cov(X,Y)=0, on dit que X et Y sont non corrélées. 
\end{Def}

\begin{theo}[Inégalité de Cauchy-Schwarz]
Soient X et Y deux v.a.r. On a : \[|\text{cov(X,Y)}|\leq \sqrt{V(X)}\sqrt{V(Y)}\]
et \[\text{cov}(X,Y)^2=V(X)V(Y) \Leftrightarrow \exists(\alpha,\beta,\gamma)\neq(0,0,0); \alpha X+\beta Y= \gamma\ p.s.\]
\end{theo}

\begin{dem}
Considérons le prolynôme P défini pour tout $\lambda\in \mathbb{R}$ par : \[P(\lambda)=V(X+\lambda Y)=V(X)+\lambda^2V(Y)+2\lambda \text{cov}(X,Y) \geq 0\]
Par conséquent : \begin{eqnarray*}
\Delta &=& 4 \text{cov}^2(X,Y) - 4V(X)V(Y) \leq 0 \\
&\Leftrightarrow& \text{cov}^2(X,Y)\leq V(X)V(Y) \\
&\Leftrightarrow& |\text{cov}(X,Y)|\leq \sqrt{V(X)}\sqrt{V(Y)}
\end{eqnarray*}

Supposons : $\exists(\alpha,\beta,\gamma)\neq(0,0,0); \alpha X+\beta Y= \gamma$ p.s. On peut supposer $\alpha\neq 0$
\begin{eqnarray*} 
X&=&\frac{\gamma}{\alpha} - \frac{\beta}{\alpha}Y \\
\text{Cov}(X,Y)&=& \text{Cov}(\frac{\gamma}{\alpha},Y) - \frac{\beta}{\alpha} \text{cov}(Y,Y) \\
&=& 0-\frac{\beta}{\alpha} V(Y) \\
\\
\Rightarrow \text{Cov}^2(X,Y)&=& \frac{\beta^2}{\alpha^2} V(Y) = V\left(\frac{\beta}{\alpha}Y \right)V(Y) \\
&=& V\left(\frac{\gamma}{\alpha} - \frac{\beta}{\alpha}Y\right)V(Y) \\
&=& V(X)V(Y)
\end{eqnarray*}

Réciproquement, si $\text{cov}(X,Y)^2=V(X)V(Y)$ alors $\Delta=0$ et P admet une racine réelle (double) $\lambda_0$

\begin{eqnarray*}
&&P(\lambda_0)=V(X+\lambda_0Y)=0 \\
&\Leftrightarrow& E\left(((X+\lambda_0Y)-E((X+\lambda_0Y))^2\right)=0 \\
&\Leftrightarrow& X+\lambda_0Y=E((X+\lambda_0Y)
\end{eqnarray*}
Autrement dit, $X+\lambda_0Y = c$ p.s.
\end{dem}

\begin{lem}
Si $X\geq 0$ p.s. tel que E(X)=0 alors X=0 p.s.
\end{lem}

\begin{dem}
D'après l'inégalité de Markov : 
\begin{eqnarray*}
& &\forall\varepsilon>0, 0\leq\mathbb{P}(X\geq \varepsilon) \leq \frac{E(X)}{\varepsilon}=0 \\
&\Rightarrow& \forall n\in \mathbb{N}^*, \mathbb{P}\left(X\geq \frac{1}{n}\right)=0 \\
&\Rightarrow& \mathbb{P}\left(\bigcup_{n\in \mathbb{N}^*} \left\{X\geq \frac{1}{n}\right\}\right)\leq \sum_{n\geq 1} \mathbb{P}\left(X\geq \frac{1}{n}\right)=0 \\
&\Rightarrow& \mathbb{P}\left(\bigcap_{n\in \mathbb{N}^*} \left\{X\leq \frac{1}{n}\right\}\right)=1=\mathbb{P}(X=0) \\
&\Rightarrow& X=0 \ p.s.
\end{eqnarray*}
\end{dem}

\begin{Prop}
\begin{enumerate}
\item cov(X,Y)=cov(Y,X)
\item cov(aX+bY,Z)=a cov(X,Z)+b cov(Y,Z), $\forall(a,b)\in \mathbb{R}^2$
\item cov(X,X)=V(X)
\item V(X+Y)=V(X)+V(Y)+2cov(X,Y)
\item X$\Inde$Y $\Rightarrow$ cov(X,Y)=0 (Réciproque fausse)
\end{enumerate}
\end{Prop}

\subsection{Les différentes lois des grands nombres}
\begin{theo}[Loi faible des grands nombres]
Soit $(X_k)_{k\geq 1}$ une suite de v.a.r. de même loi tel que $E(X_1^2)<+\infty$ et deux à deux non corrélées. Alors \[\frac{1}{n}\sum_{k=1}^n X_k \xrightarrow[n\to +\infty]{\mathbb{P}} E(X_1)\]
\end{theo}

\begin{dem}
Posons $S_n=\sum_{k=1}^n X_k$. 
\[\frac{S_n}{n} \xrightarrow[n\to +\infty]{\mathbb{P}} E(X_1) \Leftrightarrow \forall \varepsilon>0, \mathbb{P}\left(\left|\frac{S_n}{n}-E(X_1) \right|>\varepsilon\right)\xrightarrow[n\to +\infty]{}0\]
Soit $\varepsilon>0$ fixé. On a \[E(S_n)=\sum_{k=1}^n E(X_k)=nE(X_1)\]
\begin{eqnarray*}
\mathbb{P}\left(\left|\frac{S_n}{n}-E(X_1) \right|>\varepsilon\right) &=& \mathbb{P}\left(\left|S_n-nE(X_1) \right|>n\varepsilon\right) \\
&\leq& \frac{V(S_n)}{(n\varepsilon)^2}
\end{eqnarray*}
Or, \begin{eqnarray*}
V(S_n)&=&V(\sum_{k=1}^n X_k) \\
&=& \sum_{k=1}^n V(X_k)\ (car\ X_k\ 2\ à\ 2\ non\ correlées) \\
&=& nV(X_1)\ (car\ X_k\ identiquement\ distribuées)
\end{eqnarray*}
\[\Rightarrow \mathbb{P}\left(\left|\frac{S_n}{n}-E(X_1) \right|>\varepsilon\right) \leq \frac{V(X_1)}{n\varepsilon^2}\xrightarrow[n\to +\infty]{}0\]
avec $V(X_1)$ fini car $E(X_1^2)<+\infty$. D'où \[\frac{1}{n}\sum_{k=1}^n X_k \xrightarrow[n\to +\infty]{\mathbb{P}} E(X_1)\]
\end{dem}

\begin{theo}[loi forte des grands nombres - admis]
Soit $(X_k)_{k\geq 1}$ une suite de v.a.r. i.i.d.
\[E(|X_1|)<+\infty \Rightarrow \frac{1}{n}\sum_{k=1}^n X_k \xrightarrow[n\to +\infty]{p.s.} E(X_1)\]
\end{theo}

\subsection{Convergence en loi}
\begin{Def}
Soient Y et $(Y_n)_{n\geq 1}$ des v.a.r. \\
On dit que la suite $(Y_n)_{n\geq 1}$ converge en loi vers la v.a. Y si : \[F_{Y_n}(x)\xrightarrow[n\to +\infty]{}F_Y(x)\]
pour tout point de continuité $x$ de $F_Y$ (avec $F_X$ f.d.r. de la v.a. X) \\
On note alors : \[Y_n \xrightarrow[n\to +\infty]{\mathcal{L}}Y\]
\end{Def}

\begin{rmq}
\begin{itemize}
\item Convergence p.s. $\Rightarrow$ Convergence en proba $\Rightarrow$ Convergence en loi
\item La convergence en loi n'est pas stable pour la somme des variables aléatoires : 
\[Y_n \xrightarrow[n\to +\infty]{\mathcal{L}}Y\ et\ Z_n \xrightarrow[n\to +\infty]{\mathcal{L}}Z \not\Rightarrow Y_n +Z_n \xrightarrow[n\to +\infty]{\mathcal{L}}Y +Z\]
\end{itemize}
\end{rmq}

\begin{theo}[Central limit]
Soit $(X_k)_{k\geq 1}$ une suie de v.a.r., iid et de carré intégrable (ie $E(X_1^2)<+\infty$). On a alors :
\[W_n \frac{1}{\sigma \sqrt{n}} \sum_{k=1}^n (X_k-\mu) \xrightarrow[n\to +\infty]{\mathcal{L}} N\]
où $\mu = E(X_1)$, $\sigma^2=V(X_1)>0$ et $N \hookrightarrow \mathcal{N}(0,1)$
\end{theo}

On a donc : \[\forall x\in \mathbb{R}, \mathbb{P}(W_n\leq x) \xrightarrow[n\to +\infty]{} \mathbb{P}(N\leq x)=F_N(x)\]
avec \[F_n(x)=\frac{1}{\sqrt{2\pi}}\int_{-\infty}^x e^{\frac{-t^2}{2}} dt\]
Ce qui équivaut à : \[\forall a<b, \mathbb{P}(a\leq W_n \leq b) \xrightarrow[n\to +\infty]{} \sqrt{2\pi}\int_a^b e^{\frac{-t^2}{2}} dt\]

\section{Vecteurs aléatoires}
\begin{Def}
Soit X=$^t(X_1,...,X_d)$ un vecteur aléatoire de dimension $d\in\mathbb{N}$. On appelle espérance de X le vecteur \[E(X)=^t(E(X_1),...,E(X_d)\]
\end{Def}

\begin{Prop}
\begin{itemize}
\item Si X=$^t(X_1,...,X_d)$ et Y=$(Y_1,...,Y_d)$ sont deux vecteurs aléatoires de dimension d, alors \[E(X+Y)=E(X)+E(Y)\]
\item Si M est une matrice de nombres réels et si X est un vecteur aléatoire tel que MX soit bien défini, alors $E(MX)=M\times E(X)$
\item Si $\phi:\mathbb{R}^d \rightarrow \mathbb{R}$ est convexe alors $\phi(E(X))\leq E(\phi(X))$ (Inégalité de Jensen)
\end{itemize}
\end{Prop}

\begin{Def}
Soit X=$^t(X_1,...,X_d)$ tel que $X_i$ soit de carré intégrable pour tout $1\leq i\leq d$. On appelle matrice de covariance (ou matrice de dispersion) du vecteur aléatoire X, la matrice : 
\begin{eqnarray*}
V(X)&=&E((X-E(X)) ^t(X-E(X)) \\
&=& \left(cov(X_i,X_j) \right)_{1\leq i,j\leq d}
\end{eqnarray*}
\end{Def}

\begin{Prop}
\begin{enumerate}
\item Si b=$(b_1,...,b_d)$ est un vecteur constant et si X=$(X_1,...,X_d)$ est un vecteur aléatoire alors $V(X+b)=V(X)$
\item Si M est une matrices de nombres réels tel que MX soit bien défini, alors 
\begin{eqnarray*}
V(MX)&=& E((MX-E(MX)) ^t(MX-E(MX))) \\
&=&E(M(X-E(X)) ^t(M(X-E(X)))) \\
&=&E(M(X-E(X)) ^t(X-E(X)) ^tM) \\
&=& M V(X) ^tM 
\end{eqnarray*}
\item V(X) est une matrice symétrique semi-définie positive. \\
Autrement dit, pour tout vecteur Y non nul, on a $^tYV(X)Y \geq 0$, ou encore, toutes les valeurs propres de V(X) sont positives ou nulles.
\end{enumerate}
\end{Prop}

\begin{Def}[Vecteurs alétoires gaussiens]
On dit que le vecteur alétoire X=$^t(X_1,...,X_d)$ de $\mathbb{R}^d$ est gaussien si pour toute application linéaire $u:\mathbb{R}^d \rightarrow \mathbb{R}$, la variable aléatoire u(X) est une v.a. réelle gaussienne.
\end{Def}

\begin{rmq}
$\forall 1\leq i\leq d$, considérons l'application linéaire 
\begin{eqnarray*}
u_i : \mathbb{R}^d &\rightarrow& \mathbb{R}\\
(x_1,...,x_d) &\mapsto& x_i
\end{eqnarray*}
Ainsi, $X_i=u_i(X)$, avec X vecteur aléatoire gaussien. Donc, par définition, $X_i$ est une v.a. gaussienne sur $\mathbb{R}$. \\
La réciproque est fausse.
\end{rmq}

\begin{Prop}
La loi d'un vecteur aléatoire gaussien X=$^t(X_1,..,X_d)$ est caractérisée par son vecteur espérance \[m=^t(E(X_1),...,E(X_d))\] et sa matrice de dispersion 
\[\Gamma=(cov(X_i, X_j)_{1\leq i,j\leq d}\]
La loi de X est notée $\mathcal{N}_j(m,\Gamma)$
\end{Prop}

\begin{Prop}
Soient X et Y deux v.a. réelles indépendantes dont les lois admettent des densités de probabilité $f_X$ et $f_Y$ respectivement. \\
Alors la loi de Z=X+Y admet également une densité de probabilité $f_Z$ définie pour tout $x\in \mathbb{R}$ par 
\[f_Z(x)=(f_X*f_Y)(x) = \int_{\mathbb{R}} f_X(t) f_Y(x-t)dt\]
\end{Prop}

\begin{dem}
$X\Inde Y$, Z=X+Y. \\
Montrons que la loi de Z admet une densité de probabilité $f_Z$ définie pour tout $x\in \mathbb{R}$ par \[f_Z(x)=(f_X * f_Y)(x)\]
Tout d'abord, comme X et Y sont indépendantes, la loi du couple (X,Y) admet une densité de probabilité $f_{(X,Y)}$ définie pour tout $(x,y)\in \mathbb{R}^2$ par \[f_{(X,Y)}(x,y)=f_X(x) f_Y(y)\]
Soit $h:\mathbb{R} \rightarrow \mathbb{R}$ mesurable.

\begin{eqnarray*}
E(h(Z))&=&E(h(\phi(X,Y)))\ ou\ \phi(X,Y)=X+Y \\
&=& E(\tilde{h}(X,Y))\ ou\ \tilde{h}=h\circ \phi \\
&=& \int \int_{\mathbb{R}^2} \tilde{h}(x,y) f_{(X,Y)}(x,y) dx dy \\
&=& \int \int_{\mathbb{R}^2} h(x+y) f_X(x) f_Y(y) dx dy \\
&=& \int_{\mathbb{R}} \left(\int_{\mathbb{R}} h(x+y) f_X(x) dx \right) f_Y(y) dy
\end{eqnarray*}
Posons $u=x+y \Rightarrow du=dx$.
\begin{eqnarray*}
E(h(Z))&=& \int\int_{\mathbb{R}^2} h(u)f_X(u-y) f_Y(y) du dy \\
&=& \int_{\mathbb{R}} h(u) \left(\int_{\mathbb{R}} f_X(u-y) f_Y(y) dy  \right) du
\end{eqnarray*}
On a donc : \begin{eqnarray*}
f_Z(u)&=& \int_{\mathbb{R}} f_X(u-y)f_Y(y) dy \\
&=& (f_X*f_Y)(u)
\end{eqnarray*}
\end{dem}

\begin{coro}
La somme de deux v.a. gaussiennes indépendantes est encore une v.a; gaussienne.
\end{coro}

\begin{dem}
Soient $X\hookrightarrow \mathcal{N}(0,1)$ et $Y\hookrightarrow \mathcal{N}(0,1)$ tel que $X\Inde Y$ \\
Montrons que $Z=X+Y \hookrightarrow \mathcal{N}(0,2)$

\begin{eqnarray*}
f_Z(x) &=& \int_{\mathbb{R}} f_X(t) f_Y(x-t) dt \\
&=& \int_{\mathbb{R}} \frac{1}{\sqrt{2\pi}} e^{-\frac{t^2}{2}} \times \frac{1}{\sqrt{2\pi}} e^{-\frac{(x-t)^2}{2}} dt \\
&=& \frac{1}{2\pi} \int_{\mathbb{R}} e^{-\frac{t^2}{2}} e^{-\frac{1}{2}(x^2-2xt+t^2)} dt \\
&=& \frac{1}{2\pi} \int_{\mathbb{R}} e^{-(\frac{x^2}{2} -xt+t^2)} dt \\
&=& \frac{1}{2\pi} e^{-\frac{x^2}{4}} \int_{\mathbb{R}} e^{-(\frac{x^2}{4} -xt+t^2)}dt \\
&=& \frac{1}{2\pi} e^{-\frac{x^2}{4}} \int_{\mathbb{R}} e^{-(\frac{x}{2} -t)^2}dt \\
&=& \frac{1}{2\pi} e^{-\frac{x^2}{4}} \underbrace{\int_{\mathbb{R}} e^{-\frac{(t-\mu)^2}{2\sigma^2}}}_{=\sqrt{2\pi}\times \sigma}dt \qquad ou\ \mu=\frac{x}{2},\ \sigma^2=\frac{1}{2} \\
&=& \frac{1}{2\pi} e^{-\frac{x^2}{4}}\times \sqrt{2\pi}\frac{1}{\sqrt{2}} \\
&=& \frac{1}{2\sqrt{\pi}} e^{-\frac{x^2}{4}}
\end{eqnarray*}
D'où $Z\hookrightarrow \mathcal{N}(0,2)$
\end{dem}

\begin{theo}[admis]
Soit $X=(X_1,...,X_d)$ i, vecteur aléatoire gaussien de moyenne $m=(m_1,...,m_d)$ et de matrice de covariance $\Gamma$ (on note $X\hookrightarrow \mathcal{N}_d(m,\Gamma)$) \\
Si $\Gamma$ est inversible alors la loi de X admet une densité de probabilité $f_X$ définie pour tout $x=(x_1,...,x_d) \in \mathbb{R}^d$ par : 
\[f_X(x)=\frac{1}{(2\pi)^{\frac{d}{2}} \sqrt{|det(\Gamma)|}} \exp\left(-\frac{1}{2} <x-m,\Gamma^{-1} (x-m)> \right)\]
\end{theo}

\begin{rmq}
Si d=1, on retourve la densité de la loi $\mathcal{N}(m,\sigma^2)$ 
\end{rmq}

\section{Fonctions caractéristiques}
\begin{Def}
\begin{itemize}
	\item On appelle fonction caractéristique d'une v.a.r. X l'application $\phi_X$ définie sur $\mathbb{R}$ et à valeur dans le disque unité fermé du plan complexe par : 
\[\forall t\in \mathbb{R}, \phi_X(t)=E(e^{itX})=E(\cos(tX))+iE(\sin(tX))\]
	\item Si la v.a. est à valeur dans $\mathbb{R}^d$ ($d\in \mathbb{N}^*$), la fonction caractéristique (de la loi) de X, notée encore $\phi_X$, est l'application définie
sur $\mathbb{R}^d$ et à valeurs dans le disque unité fermé du plan complexe par :
\[\forall t\in \mathbb{R}, \phi_X(t)=E(e^{i<t,X>})=E(\cos(<t,X>)) +iE(\sin(<t,X>))\]
\end{itemize}
\end{Def}



\begin{rmq}
\begin{enumerate}
	\item Si $X(\Omega)\subset \mathbb{Z}$ alors \[ \forall t\in \mathbb{R}, \phi_X(t)=\sum_{k\in\mathbb{Z}}\mathbb{P}(X=k)e^{itk}\]
	\item Si la loi de X admet une densité de probabilité $f_X$ sur $\mathbb{R}^d$ ($d\in\mathbb{N}^*$) alors 
\[\forall t\in \mathbb{R}^d, \phi_X(t) =\int_{\mathbb{R}} e^{i<t,X>} f_X(t) dt\]
Il s'agit de la transformée de Fourier de $f_X$
\item Si $X\hookrightarrow \mathcal{N}(0,1)$, alors $\forall t\in \mathbb{R}, \phi_X(t)=e^{-\frac{t^2}{2}}$ \\
(Réciproque vraie)
\end{enumerate}
\end{rmq}

\paragraph{Propriétés : \\}
\begin{enumerate}
\item $\forall a,b \in \mathbb{R}$, $\forall t \in \mathbb{R}$, $\phi_{aX+b}(t)=e^{itb}\phi_X(at)$
\item Si $X(\Omega) \subset \mathbb{R}^d$, alors pour toute matrice A réelle $n\times d$ et tout matrice B réelle $n\times 1$ : 
\[\phi_{AX+B}(t)=e^{i<t,B>}\phi_X(^tAt),\ \forall t\in \mathbb{R}^d\]
\item $\phi_X(0)=1$ et $\phi_{-X}(t)=\phi_X(-t)=\overline{\phi_X(t)}$
\item Si X est une v.a.r. intégrable alors $\phi_X$ est de classe $\mathcal{C}^1$ et : 
\[\forall t\in \mathbb{R}, \phi_X '(t)=iE(Xe^{itX})\]
En particulier, si t=0, on obtient $\phi_X '(t)=iE(X)$

\bigskip
Plus généralement, si X est p-intégrable (ie $E(|X|^p)<\infty$) avec $p\in \mathbb{N}^*$, alors $\phi_X$ est de classe $\mathcal{C}^p$ et pour tout $t\in \mathbb{R}$ 
\[\phi_X^{(p)} (t)=i^p E(X^p e^{itX})\]
En particulier, $\phi_X^{(p)}(0)=i^p E(X^p)$ 

\bigskip
Si $X(\Omega)\subset \mathbb{R}^d$ et si X est d-intégrable, $\phi_X$ est de clsse $\mathcal{C}^{\alpha}$ et pour tout $p=(p_1,...,p_d) \in \mathbb{N}^d;\ p_1+...+p_d\leq \alpha$, on a :
\[\frac{\partial^{p_1+...+p_d}}{\partial t_1^{p_1}...\partial t_d^{p_d}}\phi_X(0)=i^{p_1+...+p_d} E(X_1^{p_1}...X_d^{p_d})\]

\item Si X et Y sont deux v.a.r. indépendantes : \[\phi_{X+Y}(t) = \phi_X(t)\phi_Y(t)\]
\end{enumerate}

\begin{theo}
Deux v.a.r. (ou d-dimensionnelle) ont même loi si et seulement si elles ont même fonction caractéristique.
\end{theo}

\begin{theo}
Si X est une v.a. d-dimensionnelle et si \[\int_{\mathbb{R}^d} |\phi_X(t)| dt_1...dt_d <\infty\]
alors X admet une densité de probabilité $f_X$ continue sur $\mathbb{R}^d$ définie pour tout $x\in \mathbb{R}^d$ par :
\[f_X(x)=\frac{1}{(2\pi)^d} \int_{\mathbb{R}^d} \phi_X(t) e^{-i<t,X>} dt\]
\end{theo}

\paragraph{Transformée de Laplace d'une v.a.r. positive : \\}
Lorsque X est positive p.s., on peut utiliser la transformée de Laplace.

\begin{Def}
Soit X une v.a.r. positive p.s. On appelle transformée de Laplace de la loi de X la fonction :
\begin{eqnarray*}
L_X : \mathbb{R}^+ &\to& ]0,1] \\
\lambda &\mapsto& E(e^{-\lambda X})
\end{eqnarray*}
\end{Def}

\section{Conditionnement d'une variable aléatoire, espérance conditionnelle}
Soient X et Y deux v.a. réelles définies sur un espace probabilisé $(\Omega, \mathcal{F}, \mathbb{P})$.
\subsection{Conditionnement d'une v.a. par rapport à une autre}
\begin{theo}[de Doob]
Il existe une application :
\begin{eqnarray*}
q : \mathcal{B}(\mathbb{R})\times\mathbb{R} &\to& [0,1] \\
		(B,x) 	 &\mapsto& q(B,x)
\end{eqnarray*}
vérifiant : \begin{enumerate}
\item Pour tout $B\in \mathcal{B}(\mathbb{R})$, l'application $q(B,\bullet)$ est mesurable.
\item $\forall x\in \mathbb{R}$, l'application $q(\bullet,x)$ est une probabilité sur $(\mathbb{R},\mathcal{B}(\mathbb{R}))$
\item Pour tout $A\in\mathcal{B}(\mathbb{R}^2)$ on a : 
\begin{eqnarray*}
\mu_{(X,Y)}(A) &=& E(1_{A}(X,Y)) \\
	&=& \iint_{\mathbb{R}^2} 1_A(x,y) q(dy,x) d\mu_X(x) 
\end{eqnarray*}
avec $\mu_{(X,Y)}$ et $\mu_X$ les lois de probabilité du vecteur aléatoire (X,Y) et de la variable aléatoire X respectivement.
\end{enumerate}
\end{theo}

\begin{Def}
L'application q est appellée "loi conditionnelle de Y sachant X"
\end{Def}

\noindent \textbf{Conséquence :} \\
Pour toute fonction $\mu_{(X,Y)}$-intégrable $f:\mathbb{R}^2 \to \mathbb{R}$ (ie $\iint_{\mathbb{R}^2} |f(x,y)| d\mu_{(X,Y)}(x,y) <\infty$) on a :
\[E(f(X,Y)) = \iint_{\mathbb{R}^2} f(x,y) q(dy,x) d\mu_X(x)\]

\begin{rmq}
On montre que q du théorème ci-dessus est unique dans le sens suivant : \\
Si $\tilde{q}$ est une autre loi conditionnelle de Y sachant X alors il existe un borelien $\mu_X$-négligeable N de $\mathbb{R}$ (ie : $N\in\mathcal{B}(\mathbb{R})$ et $\mu_X(N)=0$) tel que : 
\[\forall x\in\mathbb{R}\backslash N,\ \forall B\in\mathcal{B}(\mathbb{R}),\ q(B,x)=\tilde{q}(B,x)\]
\end{rmq}

\begin{Def}
Pour tout $x\in\mathbb{R}$ on note \[q(\bullet,x)=\mathbb{P}(\bullet|X=x)\]
et $q(\bullet,x)$ est appelée "loi conditionnelle de Y sachant X".
\end{Def}

\noindent\textbf{Attention :}
Pour tout $B\in\mathcal{B}(\mathbb{R})$, $\mathbb{P}(B|X=x)=q(B,\bullet)$ est classe d'équivalence par l'égalité $\mu_X$-p.s. de fonctions mesuablres de $\mathbb{R}$ dans [0,1].

Par conséquent, pour un $x\in\mathbb{R}$ particulier tel que $\mu_X(\{x\})=0$ (ie $\mathbb{P}(X=x)=0$), l'expression $\mathbb{P}(B|X=x)$ n'a pas de sens.

On détermine en général q par identification à l'aide des points 1, 2 et 3 du théorème de Doob. Cependant, il y a au moins 3 cas où l'on a un résultat explcite :

\bigskip
\textbf{1er cas :} Si $\mu_X$ est discrète alors pour tout $x\in \mathbb{R}$ tel que $\mathbb{P}(X=x)>0$ alors : 
\[q(B,x)=\mathbb{P}(B|X=x)=\frac{\mathbb{P}(B\cap\{X=x\})}{\mathbb{P}(X=x)}\]

\bigskip
\textbf{2eme cas :} Si le couple de v.a. (X,Y) admet une densité de probabilité $f_{(X,Y)}$ alors, pour $\mu_X$-presque tout $x\in\mathbb{R}$ tel que $f_X(x)\neq0$ : 
\[\forall B\in \mathcal{B}(\mathbb{R}), q(B,x)=\int_B \frac{f_{(X,Y)}(x,y)}{f_X(x)}dy\]

\bigskip
\textbf{3eme cas :} Si les v.a. X et Y sont indépendantes, alors, pour $\mu_X$-presque tout $x\in\mathbb{R}$, on a $q(\bullet,x)=\mu_Y$ \\
(ie la loi conditionnelle de Y sachant $X=x$ ne dépend pas de $x$)

\subsection{Espérance conditionnelle de Y sachant X}
On suppose que Y est intégrable. On montre à l'aide du théorème de Radon-Nikodyn qu'il existe une unique classe d'équivalence pour l'égalité $\mathbb{P}$-p.s. de la v.a. Z à valeur dans $\mathbb{R}$ vérifiant : \begin{enumerate}
\item Z est $\sigma(X)$-mesurable \\ ie : $\forall A \in \mathcal{B}(\mathbb{R}),\ Z^{-1}(A) \in \sigma(X)=X^{-1}(\mathcal{B}(\mathbb{R}))$
\item $\forall A\in \sigma(X)$ : \[\int_A Zd\mathbb{P}=\int_A Yd\mathbb{P}\]
\end{enumerate}

\begin{Def}
La classe d'équivalence de v.a. Z ainsi définie est appelée "espérance conditionnelle de Y sachant X" (ou encore "epsérance conditionnelle de Y sachant la tribu $\sigma(X)$").
\\ Elle est notée E(Y|X).
\end{Def}

On détermine E(Y|X) par identification à l'aide de 1 et 2. \\
D'autre part, on peut vérifier que si q est la loi conditionnelle de Y sachant X, alors l'application : 
\begin{eqnarray*}
\Omega &\to& \mathbb{R} \\
\omega &\mapsto& \int_{\mathbb{R}} y\ q(dy,X(\omega))
\end{eqnarray*}
est une version de E(Y|X).

Plus généralement, pour toute application mesurable $f$ tel que $f(Y)$ soit intégrable $\omega \mapsto \int_{\mathbb{R}} f(y)\ q(dy,X(\omega))$ est une version de E(f(Y)|X)

\begin{rmq}
E(Y|X) est une fonction $\sigma(X)$-mesurable. D'après un (autre) théorème de Doob, il existe une fonction $\phi : \mathbb{R} \to \mathbb{R}$ mesurable tel que : 
\[E(X|Y)=\phi(X)\]
$\phi$ est notée $\phi(x)=E(Y|X=x)$ mais il faut faire attention au fait que $\phi$ n'est définie que modulo l'égalité $\mu_X$-ps.  \\
On a alors pour tout $h:\mathbb{R}\to \mathbb{R}$ mesurable : 
\[E(h(Y)|X=x) = \int_{\mathbb{R}} h(y)\ q(dy,x)\]
pour $\mu_X$-presque tout $x\in\mathbb{R}$
\end{rmq}

\begin{rmq}
Pour tout $B\in \mathcal{B}(\mathbb{R})$ :
\[E(1_B(Y)|X=x) = q(B,x) = \mathbb{P}(Y\in B|X=x)\]
pour $\mu_X$-presque tout $x\in\mathbb{R}$
\end{rmq}


\end{document}
