\section{Espaces probabilisés généraux}
\begin{Def}
Soit $\Omega$ un ensemble quelconque non vide.
Une tribu $\mathfrak{F}$ sur $\Omega$ est un ensemble de parties de $\Omega$ vérifiant :
\begin{enumerate}
\item $\Omega \in \mathfrak{F}$
\item $\mathfrak{F}$ stable par complémentaire
\item $\mathfrak{F}$ stable par union dénombrable
\end{enumerate}
Le couple $(\Omega, \mathfrak{F})$ est appelé espace probabilisable. Un élément de $\mathfrak{F}$ est appelé évenement.
\end{Def}

\begin{Def}
On appelle tribu borélienne de $\mathbb{R}^d$ et on note $\mathfrak{B}(\mathbb{R}^d)$ la tribu engendrée par l'ensemble des ouverts de $\mathbb{R}^d$, c'est-à-dire la plus petite tribu qui contient l'ensemble des ouverts de $\mathbb{R}^d$.
\end{Def}

\begin{Def}
On appelle mesure de probabilité ou loi de probabilité sur l'espace mesurable (ou probabilisable) $(\Omega, F)$ une application définie sur $\mathcal{F}$ et à valurs dans $\mathbb{R}^+$ tel que :
\begin{enumerate}
\item $\mathbb{P}(A) \in [0,1] \forall A \in \mathcal{F}$
\item $\mathbb{P}(\Omega)=1$
\item Pour toute famille $\mathcal{A}$ dénombrable d'éléments deux à deux incompatibles de $\mathcal{F}$, on a \[\mathbb{P}(\bigcup_{A \in \mathcal{A}} A) = \sum_{A \in \mathcal{A}} \mathbb{P}(A)\]
\end{enumerate}
\end{Def}

\begin{Def}
On appelle densité de probabilité (continue) sur $\mathbb{R}$ (ou sur $\mathbb{R}^d$) une fonction f continue par morceaux à valeurs positives et tel que \[\int_{\mathbb{R}}f(x)dx=1\] (ou $\int_{\mathbb{R}^d}f(x)dx=1$).
\end{Def}

\begin{theo}
Si f est une densité de probabilité sur $\mathbb{R}$ (ou $\mathbb{R}^d$) alors il existe une unique mesure de probabilité $\mathbb{P}$ sur ($\mathbb{R}$, $\mathfrak{B}(\mathbb{R})$) (ou sur ($\mathbb{R}^d$, $\mathfrak{B}(\mathbb{R}^d)$) tel que pour tout intervalle ]a,b[ (ou tout cylindre $]a_1,b_1[\times...\times]a_n,b_n[$) on ait : 
\[\mathbb{P}(]a,b[)=\int_a^b f(x)dx\ ou \  \mathbb{P}(]a_1,b_1[\times...\times]a_n,b_n[)=\int_{a_1}^{b_1}...\int_{a_n}^{b_n} f(x)dx\]
\end{theo}

\addcontentsline{toc}{subsubsection}{Exemples de lois continues}
\subsection*{Exemples de lois continues}
\paragraph{Loi normale $\mathcal{N}$(0,1)}
X$\hookrightarrow \mathcal{N}(0,1)$ \\
\begin{eqnarray*}
f_X(x)&=&\frac{1}{\sqrt{2\pi}} e^{\frac{-x^2}{2}} \\
F_X(x)&=&\int_{-\infty}^{+\infty} f_X(t)dt \\
\mathbb{P}(a \leq X \leq b) &=& \int_{a}^{b} f_X(t)dt
\end{eqnarray*}

\paragraph{Loi normale $\mathcal{N}(\mu,\sigma^2)$}
X$\hookrightarrow \mathcal{N}(\mu,\sigma^2)$ \\
\[f_X(x)=\frac{1}{\sqrt{2\pi}\sigma} e^{\frac{-(x-\mu)^2}{2\sigma^2}}\]

\paragraph{Loi exponentielle $\mathcal{E}(\lambda)$}
X$\hookrightarrow \mathcal{E}(\lambda)$ \\
\[f_X(x)=\lambda e^{-\lambda x}\ 1_{\mathbb{R}_+^*}(x)\]

\paragraph{Loi uniforme sur l'intervalle ]a,b[}
X$\hookrightarrow \mathcal{U}(]a,b[)$ \\
\[f_X(x)=\frac{1}{b-a} \ 1_{]a,b[}(x)\]

\paragraph{Loi uniforme sur un borélien $\mathfrak{B}$ de $\mathbb{R}^2$}
X$\hookrightarrow \mathcal{U}(B)$ avec $B \in \mathfrak{B}(\mathbb{R}^2)$ fixé\\
$\forall A \in \mathfrak{B}(\mathbb{R}^2)$, $\mathbb{P}(X\in A)=\frac{\lambda_2 (A \cap B)}{\lambda_2(B)}$
\[f_X(x)=\frac{1}{\lambda_2(B)} \ 1_{B}(x)\]

Autres lois : Gamma, Bêta, de Student, du $\chi^2$, de Fisher...

\begin{Def}
On appelle v.a. réelle (ou d-dimensionnelle) à valeur dans un borélien E une application mesurable X définie sur $\Omega$ à valeurs dans E, ie \[\forall A \in \mathfrak{B}(E)\{x \in A\}=X^{-1}(A) \in \mathcal{F}\]
Si E=$\mathbb{R}$, on dit que X est une v.a. réelle \\
Si E=$\mathbb{R}^d$, on dit que X est un vecteur aléatoire réel
\end{Def}

\begin{Prop}
\begin{enumerate}
\item La somme et le produit d'un nombre fini de v.a.r. est une v.a.r.
\item Si $(X_n)_{n \geq 1}$ est une suite de v.a.r. tel que $X_n(\omega) \xrightarrow[n \rightarrow +\infty]{} X(\omega)$ pour tout $\omega \in \Omega$ alors l'application X est une v.a.r
\end{enumerate}
\end{Prop}

\begin{Def}
Soit X une v.a.r. à valeur dans $E\in \mathfrak{B}(\mathbb{R})$. L'application Q définie pour tout $B\in \mathfrak{B}(\mathbb{R})$ par \[Q(B)=\mathbb{P}(X\in B)\] est une mesure de probabilité. On l'appelle loi de probabilité sur E de la v.a. X.
\end{Def}

\paragraph{Loi exponentielle}\
Soit $\lambda$>0. On dit qu'une var X suit la loi exponentielle de paramètre $\lambda$>0 si \[\forall 0<a<b, \mathbb{P}(X\in[a,b])=\int_a^b \lambda e^{-\lambda x}\]
On note alors $X \hookrightarrow \mathcal{E}(\lambda)$

\paragraph{Remarque} : $\mathbb{P}(X\in ]0,+\infty[)=1$ et donc nécessairement, $\mathbb{P}(X\leq 0)=0$.

\begin{Prop}
Si $X\hookrightarrow \mathcal{E}(\lambda)$ avec $\lambda$ >0, alors $aX \hookrightarrow \mathcal{E}(\frac{\lambda}{a})$, $\forall a>0$.
En particulier, $\lambda X \hookrightarrow \mathcal{E}(1)$
\end{Prop}

\begin{dem}
La fonction de répartition de $\mathcal{E}(\lambda)$ est \[F_X(x) = \left( 1-e^{-\lambda x} \right) 1_{\mathbb{R}_+^*} (x)\]
Posons Y=aX. \\
\begin{eqnarray*}
F_Y(x) &=& \mathbb{P}(Y \leq x) \\
&=& \mathbb{P} \left( X \leq \frac{x}{a} \right) \\
&=& F_X \left( \frac{x}{a} \right) \\
&=& \left( 1-e^{-\frac{\lambda}{a} x} \right) 1_{\mathbb{R}_+^*} (x)
\end{eqnarray*}
Il s'agit de la fonction de répartition de la loi $\mathcal{E}(\frac{\lambda}{a})$
\end{dem}

\begin{Prop}
Pour rappel, [X]=E(X) (partie entière de X). \\
Si $X\hookrightarrow \mathcal{E}(\theta)$, alors $[X]+1 \hookrightarrow \mathcal{G}(1-e^{-\theta})$
\end{Prop}

\begin{dem}
On pose Z=[X]+1. On a $Z(\Omega)=\mathbb{N}^*$. \\
Soit $k\in \mathbb{N}^*$.
\begin{eqnarray*}
\mathbb{P}(Z=k) &=& \mathbb{P}([X]=k-1) \\
&=& \mathbb{P}(k-1 \leq X < k) \\
&=& \int_{k-1}^k \theta e^{-\theta x} dx \\
&=& \left[ -e^{-\theta x} \right]_{k-1}^k \\
&=& e^{-(k-1) \theta} - e^{-k \theta} \\
&=& e^{-\theta (k-1)} \left( 1-e^{- \theta} \right)
\end{eqnarray*}
En posant $p=1-e^{-\theta}$, $\mathbb{P}(Z=k) = (1-p)^{k-1} p$. \\
Donc $[X]+1 \hookrightarrow \mathcal{G}(1-e^{-\theta})$.
\end{dem}

\begin{Prop}
De la même manière, on peut montrer que : \\
Soit $X \hookrightarrow \mathcal{E}(\theta)$ et $(N_n)_{n\geq 0} \in \left( \mathbb{R}_+^* \right)^{\mathbb{N}}$ tel que $N_n \xrightarrow[n \rightarrow +\infty]{} +\infty$.
\begin{enumerate}
\item Pour tout $n \geq 0$, $[N_n X] \hookrightarrow \mathcal{G}\left( 1-e^{-\frac{\theta}{N_n}} \right)$ 
\item $\lim_{n \rightarrow +\infty} \frac{[N_n X]}{N_n} = X$ de façon presque sûr (ie $\mathbb{P} \left( \left\{ \omega \in \Omega | \lim_{n\rightarrow +\infty} \frac{[N_n X](\omega)}{N_n}=X(\omega) \right\} \right) =1$)
\end{enumerate}
\end{Prop}

\begin{theo}
Soit X une v.a. à valeurs dans $\mathbb{R}^+$. Les propriétés suivantes sont équivalentes :
\begin{enumerate}
\item X est une v.a. de loi exponentielle
\item Pour tout réel t>0, $\mathbb{P}(X>t) \neq 0$ et \[\forall s>0, \mathbb{P}(X>t+s | X>t) = P(X>s)\] (Propriété sans mémoire)
\end{enumerate}
\end{theo}

\begin{dem}[de la première implication]
t>0 et s >0 \\
\begin{eqnarray*}
\mathbb{P}(X>t+S | X>t) &=& \frac{\mathbb{P}\left( (X>t+s) \cap (X>t) \right)}{\mathbb{P}(X>t)} = \frac{\mathbb{P}(X>t+s)}{\mathbb{P}(X>t)} \\
&=& \frac{1-F_X(s+t)}{1-F_X(x)} \\
&=& \frac{e^{-\theta (s+t)}}{e^-{\theta t}}=e^{-\theta s} \\
&=& 1-F_X(s) \\
&=& \mathbb{P}(X>s)
\end{eqnarray*}
\end{dem}

\begin{theo}
Soit X une v.a. à valeurs dans $\mathbb{N}^*$. Il t a équivalence entre :
\begin{enumerate}
\item X est une v.a. de loi géométrique
\item $\forall n\in \mathbb{N}$, $\mathbb{P}(X>n) \neq 0$, et : \[\forall k\in \mathbb{N}, \mathbb{P}(X>n+k | X>n) = \mathbb{P}(X>k)\]
\end{enumerate}
\end{theo}
